\chapter{Planning}
\section{How to avoid bad planning}
In this section bad planning will be described, as a number of significant consequences follows if different tasks and jobs are poorly planned. Tools to overcome bad planning will also be described.

Bad planning can have different consequences resulting in outcomes such as, plane delays, lost income, stress, etc\cite{BadPlan_CEN12LA007}. It is therefore important that the planning is as efficient as possible, to ensure that the ground handling companies can perform all there tasks in the desired time frame.

To avoid bad planning, it is important to utilize different tools, those  tools is described in \textbf{"INSERT REFERENCE"}. Avoiding bad planning does not necessarily suggest that one would have to make adjustments to the current way of handling different operations, however it could be necessary that one would need to come up with new ideas on how to avoid bad planning.


Tom Mochal who is president of TenStep\cite{AvoidP_TenStep} says that one of the major things that can make projects fail, or take longer time than needed, is the way that not all job are defined good enough. One way it could be made more clear to the people working on the project or job, is to give them a PDA, smarthone or tablet, and write out the job on the device. This would make sure that they have understood the job properly\cite{AvoidPlan_smart}.


Another method that could be used, would be to give the employees the first part of a day to make sure that they had properly understood all the tasks that they are to do that day.

In conclusion; bad planning should be avoided in order to minimize plane delays, lost income and stress. An electronic device could help the employees to get a better understanding of what they have to at the job. 