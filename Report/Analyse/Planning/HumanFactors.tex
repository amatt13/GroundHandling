\section{Aircraft ground handling and human factors}
In NLR Air Transport Safety Institute's [insert relation to European Commercial Aviation Safety Team] report, "Aircraft ground handling and human factors" finished in April 2010, on the; "... causal factors which lead to human errors during the ground handling process and create unsafe situations, personal accidents or incidents." (Page 1 of the report); it was found that the largest safety related issues according to operational personnel and management comes from standardization of phraseology on the ramp and human factors such as time pressure, stress, fatigue and communication. This part of the report will describe the detailed findings of the query related to this project and their recommendations to the ground handling companies.

First of all, one of the interesting findings, in the report, was that of what types of accidents happen, and at what relative frequency. In figure \ref{FrequencyOfIncidents}, we can clearly see that incidents that cause operational disruptions, equipment damage and aircraft damage happen at least once a week. As described in section [Freight] damage will not only be costly to repair, it will also most likely cause airplane delays, which can be very expensive for the airlines. Of course operational disrupts also result in delays and therefore loss of income.

\begin{figure}[!h]
\centering
\includegraphics[width=\textwidth]{Grafik/FrequencyOfIncidents}
\caption{The frequency of how often incidents of different categories happens.}
\label{FrequencyOfIncidents}
\end{figure}

To underline this point the study found that delay of incoming and departing flights as a cause of these incidents each happens around once a week (page 29).

[Somebody please write about the "`Direct Causes on page 34 of the report, i have no clue!]

Furthermore the survey also researched the contributing factors and how often they contribute to the different kind of accidents. As seen i figure \ref{ContributingFactors} it found that the two most contributing factors are personal and communication, i.e. mistakes made by people and errors in the communication between people.

\begin{figure}
\centering
\includegraphics[width=\textwidth]{Grafik/ContributingFactors}
\caption{The contributing factors and how often they contribute to accidents.}
\label{ContributingFactors}
\end{figure}

"Next to Personal factorsand Communication, Management attributes a relative high frequency to the contributing factors of Environment/facilities/rampand Leadership/supervision. With regard to the Environment/facilities/ramp, it may prove to be difficult to mitigate risks resulting from human factors, since aspects related to the environment, facilities and the ramp are mostly managed by the Airport Authorities. With regard to Leadership and supervisionis Management apparently aware that poor leadership or supervision may easily lead to human errors or incident.

Operational personnel provide a higher frequency than Management to the contributing factor of Equipment/tools/safety equipment. This is probably caused by their daily, hands-on experience with the equipment and tools. When compared to the direct causes presented in figure 9, Equipment/tools/safety equipmenthas dropped to the third place, although Operational personnel provide an almost similar frequency to the first three factors in figure 10. For Management, Equipment/tools/safety equipmentdrops to the fifth place. This is due to the fact that there are numerous ways in which equipment or tools may contribute to incidents or human errors."

A very important conclusion, related to our project, in the report is which factors contribute to errors and mistakes made by operational personnel and management. As seen in figure \ref{PersonalFactors} it was found that the three major factors contributing to errors and mistakes made by the personnel is time pressure, stress, fatigue and also, very relevant to our project, motivation has a high rating. Especially time pressure is a very high contributor according both operational personnel and management, also because it is expected that both stress and fatigue most likely is a consequence of time pressure. In interviews made with both management and the operational personnel it was found that the reason for fatigue is most likely caused by the ground handling staff having to work double shifts (for different employers)to generate sufficient income. Also these interviews expressed that professional pride to meet the departure time may result in shortcuts have to be taken.

\begin{figure}
\centering
\includegraphics[width=\textwidth]{Grafik/PersonalFactors}
\caption{Breakdown of channels used to book flights.}
\label{PersonalFactors}
\end{figure}

Therefore it is very important to take these contributors into account when considering a solution to prevent and solve the problem which is damage to aircraft, equipment, personal injury, operational disrupts and environmental impact.

"The safety culture of GSP plays an important role in the correct management of time pressure. From the safety culture assessments it was determined that within the safety culture characteristic Awareness, the indicator Attention for safety provided the lowest rating for all participating GSP. This related partially to whether the primary concern is to worksafely or to meet the scheduled departure time. In one of the interviews it was told that Operational personnel often see safety and a fast turnaround to meet the OTD as incompatible, whereas in reality there is always a balance between safety and speed. This balance may differ for each turnaround due to the dynamic environment or different conditions, but when the right balance is found, safety is not compromised."

"Communication between staff and between departments is considered by both Management and Operational personnel as human factors that may contribute to errors. Operational personnel also provide a high frequency for communication between ramp personnel and supervisors, and between supervisors and management. 

Communication of safety issues through the various levels of a GSP is considered important, since is raises the awareness of the role safety plays in the organisation. Communication of safety information makes it possible to learn from safety occurrences and to take proactive action. It is therefore important to promote the development and use of a safety reporting system. This was also one of the findings in the safety culture assessments, in which the safety reporting system was not known or not recognized as such by both Management and Operational personnel."

\begin{figure}
\centering
\includegraphics[width=\textwidth]{Grafik/ContributingFactors}
\caption{What kind of miscommunication causes incidents and errors.}
\label{ContributingFactors}
\end{figure}

(Page 39-47 still missing to be explained.)