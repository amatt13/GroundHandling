\section{Luggage}
Luggage is loaded on the plane using tugs, which transport containers with luggage. The Boeing 747 has seats for 416 passengers (http://alturl.com/w7bfj) and can carry roughly 6,500 kg of luggage. %Nedenstående kommer ude af sammenhæng - eller det er MEGET svært at vide hvad der menes
9,568 kg if plane would be entirely booked and every passenger had a 23 kg checked luggage, and in this estimate, the hand luggage is not taken into account. To transport such an amount of luggage, tied planning and careful transport is necessary in order to bring the luggage on the airplane in a timely fashion. %Ovenover - "tied planning", hvad menes?


Novia and SAS Ground Handling are two ground handling companies that have the responsibility of loading luggage(http://alturl.com/y9jcc). If a passenger's luggage is, by mistake, sent with a wrong plane, the passenger can contact the airline company, and then they will talk with the ground handling company, that handled the luggage. In Aalborg Lufthavn, luggage is equipped with a RFID chip that allows the airport to track the luggage, so as to make it easier to locate lost luggage.


Freight is a big part of aerial transportation and is an industry that existed as long as passenger transportation. although a lot of people have the notion that most freight is transported in airplanes for themselves but actually more than 60\% of all freight is transported taken on in the passenger flights in the unused space by passenger luggage.

"A growing proportion of world trade is now taken by air, and an increasing proportion of that is performed by the integrated, or express, carriers. These carriers offer an integrated door-todoor service with a guaranteed pickup and delivery time, relieving the customer of any direct involvement with customs and providing a tracking capability that allows the customer and the government inspection services to keep track of the shipment throughout the process. They mostly take shipments of less than 100 kg, but each of the four major players offers truly global coverage through intercontinental gateways, sub-hubs and fleets of delivery vans. Fedex, for example, delivers 3.2 million packages per day through 50,000 drop-off locations in more than 220 countries, using 671 aircraft, 41,000 vans and 138,000 employees."


"It is quite common for integrators to use space on combination carriers and vice-versa. There are also airlines that specialise in heavy lift, using small fleets of unique aircraft like the AN 124 or the Mil 10 helicopter."


"The trends also tend to reduce the ratio of value to weight, but the aircraft loads are still generally more limited by volumetric capacity than by weight limits."


"At airports in South East England, including Heathrow, each worker moves 67 tonnes per year compared with 22 tonnes in the Midlands airports [65]."


"Time in flight and in transit is most important, a saving of one hour perhaps being worth \$1000 in airport fees."


How to normally load:
"The ULDs are taken from the terminal on flat roller-bed dollys and loaded onto the aircraft main deck or belly holds using high-loader (Hi-Lo) vehicles fitted with driven roller beds.
Once inside the aircraft, they are transferred to roller beds on the floor of the aircraft, having been loaded in the correct order to achieve the necessary balance. Bulk cargo is loaded manually into the belly holds, having been brought in carts to the apron by tug and transferred to the aircraft door by self-powered conveyors (see Figure 11–4).

The specialized cargo handling equipment (e.g. Hi-Lo, see Figure 11–3) is very expensive and needs trained drivers. A careful balance needs to be struck between saving by sharing the equipment with the passenger apron or with other handling agents, and the possibility of letting the level of service slip below acceptable levels.
There needs to be enough artificial light on the apron to read documents, labels and placards, for the safety of the ground personnel while working in the midst of moving ground vehicles and aircraft servicing activity. However, it should not be so bright as to make it difficult for the flight crew to maneuver the aircraft.

There is a lot of movement of cargo between the passenger and cargo aprons at most airports.
It is therefore good practice for the distance to be kept as short as possible, within the framework of the airport’s master plan. An airside road connection of sufficient capacity should be provided, with the capability support 10 tonnes per axle, 12 metres wide, a maximum of 4\% gradient and a minimum turn radius of 20 m."

%TERMINAL DESIGN AND OPERATING CONSIDERATIONS side 223