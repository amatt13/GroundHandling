\chapter{Problem Statement}

Ground handling companies often hire low-paid workers, who work in an environment where they are exposed to congestion, stress, noise, jet-blast, extreme weather conditions and sometimes low visibility. Stress is a very factor in the work in an airport, especially to the ground handlers, since airline companies only make money, while the flights are airborne, therefore the ground handlers are very pressed on time, to reduce the time the flight spends on the ground. In many places it is also the ground handlers, who are responsible for delays; in case of a delay, they might even be deducted in salary.

When a worker is stressed he is more likely to make mistakes, which could lead to serious accidents. These accidents can first and foremost become dangerous for the workers because they can be hurt as a result of an accident. A survey made by ACI[citation needed] in 2004 showed that out of 15,119,020 aircraft movements 3,233 had accidents, concluding that 0.214\% of all turnovers had accidents.

Accidents do not only lead to dangerous situations for the workers, but they can also get very expensive for the companies; first of all because of the cost of the repair, but also because the aeroplane will then have to spend more time on the ground.
\section{Problem Formulation}
\begin{center}
\textit{Human errors and accidents during a turnaround is a result of stressed and unmotivated employees, causing delays, damage to equipment, loss of airtime and other unwanted annoyances for airlines and ground handling providers. Is it possible to formulate a software solution to resolve these issues by dynamically adapting to non-scheduled situations?}
\end{center}
